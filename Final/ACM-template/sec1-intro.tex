\section{Introduction}
\label{sec-intro}

Pattern search problem: Given that text $T[0\ldots n{-}1]$ exists over some alphabet of size $\sigma$ = $\mid$$\Sigma$$\mid$
and a pattern $P[0\ldots m{-}1]$, locate the occurrences of $P$ in $T$. $grep$ is one of the simplest forms of document retrieval 
and allows wildcard marching through regular expressions. However, it may not be ideal for large documents, ranked retrieval and flexible matching operations.

Design of a search engine should include: 
\begin{itemize}
  \item Query resolution effectiveness
  \item Query term proximity 
  \item Fast resolution of queries 
  \item Minimal usage of hardware resources 
  \item Scalability in terms of data
  \item Accommodate changes within documents
  \item Features such as Boolean restriction and phrase querying
\end{itemize}

Indexing strategies are data structures that map terms to the documents containing them for faster query retrieval.
\iffalse

\texttt Similarity measures may be defined as follows: 
\begin{itemize}
  \item $f$$_{d,t}$, the frequency of term t in document d;
  \iffalse
  \item f$_{q,t}$, the frequency of term t in the query;
  \item f$_{t}$, the number of documents containing one or more occurrences of term t;
  \item F$_{t}$, the number of occurrences of term t in the collection;
  \fi 
  \item $N$, the number of documents in the collection; and
  \item $n$, the number of indexed terms in the collection.
\end{itemize}
\fi
Inverse document frequency and term frequency statistics may be used to measure the how important a term is to a document.




